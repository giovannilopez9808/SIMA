\section{Results}
The efficiency of data availability in the four years was very different for PM\textsubscript{10} and PM\textsubscript{2.5}; with 96.9\% and 27.3\% respectively. PM\textsubscript{2.5} wasn’t measured for more than two years in the middle of the time period that was taken for the study, and that is why PM\textsubscript{10} is used to compare it with AOD.\\

The behavior of PM\textsubscript{10} and PM\textsubscript{2.5} was observed both during the year and the day. We found  there is an annual trend (Fig. 2),in the driest months (December to February) the PM levels are the highest, whereas in the month where maximum precipitation is reached (September) PM is at its minimum. This can also be seen in Figure 3, where we can see the behaviour during an average day of each season. It is in winter when the average day for the four years far exceeds the Mexican norm standard of PM\textsubscript{10} in the peak hours of activity. It should also be noted that the peaks in July of both 2015 and 2018 are due to sandstorms that came from the Sahara desert.\\

The analysis of the average behavior during a day allows us to observe the effect that human activity has on pollution. In Figure 3 and 4, the maximums of the curves correspond with the automobile traffic activity and work hours in the city. When making the distinction by weekday (Fig. 4) it can be noted that Sunday has the least concentration of PM, Thursday has the mornings maximum and Saturday is the afternoons. This accurately reflects the weekly anthropogenic activity in the AMM.