La profundidad óptica de aerosol a 550nm (AOD550nm) es una medida empleada para conocer la cantidad de partículas suspendidas en la atmósfera, desde una dada altura hasta la superficie terrestre. El material particulado de tamaño menor o igual a 10 micrómetros se denomina PM\textsubscript{10} y su concentración es medida a nivel del suelo. Estas partículas interactúan con la radiación solar causando efectos de dispersión, scattering y absorción provocando así una disminución de intensidad con respecto a la original. Con la ayuda del modelo SMARTS (Simple Model of the Atmospheric Radiative Transfer of Sunshine) y mediciones de irradiancia solar VIS+NIR (285-2800nm) del Sistema Integral de Monitoreo Ambiental (SIMA) del Estado de Nuevo León, se estimaron los valores del AOD\textsubscript{550nm} para días de cielo despejado en el periodo 2016-2020. Estos resultados fueron comparados con mediciones de AOD\textsubscript{550nm} (Collection 6.1) realizadas con el instrumento satelital MODIS-NASA para las mismas fechas, sobre el Área Metropolitana de Monterrey. Los valores del AOD\textsubscript{550nm} derivado del modelo tienen mínimos en invierno y máximos en verano, en un rango entre 0.1 y 0.6 respectivamente. Mientras que en el caso satelital tiene el mismo comportamiento con máximos alcanzando el valor de 0.8. Además estos valores fueron correlacionados con mediciones de PM\textsubscript{10} realizadas por las  mismas estaciones meteorológicas del SIMA. Se discute el origen de las fuentes de aerosol y la correlación entre ambos resultados.