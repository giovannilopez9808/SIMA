\section{Introduction}
Particulate matter (PM), or aerosol, is the set of suspended liquid or solid particles found in the atmosphere. The most relevant to human health are the ones with a diameter smaller than 10 $\mu$m (PM\textsubscript{10}) and smaller than 2.5 $\mu$m (PM\textsubscript{2.5}). With these sizes PM is capable of entering our respiratory and circulatory system, causing cardiovascular, respiratory and pregnancy problems, and even can increase the risk of pulmonary cancer and mortality \cite{Mukherjee2017}. PM can also affect visibility, cloud formation and produce damage to ecosystems and cultural sites \cite{von2015}.\\

The Monterrey Metropolitan Area (MMA) has a severe air pollution problem, with an average annual concentration of PM higher than the Official Mexican Standard (NOM) in all stations since 2000 \cite{martinez2016}. The ``Sistema de Monitoreo Ambiental” (SIMA) is the institute that since 1992 reports and shares data on air pollution over the MMA from its current 13 monitoring stations. According to its 2019 air quality report, more than half of the year the PM\textsubscript{10} standard is exceeded, so it can be concluded that it is the main pollutant in the area. The same conclusion that can be reached with the results of Benítez-García \cite{benitez2014} and the National Institute of Ecology and Climate Change \cite{inecc2011,inecc2019}. Meanwhile, PM\textsubscript{2.5} has data recording problems in most of the monitoring stations, even though it is known it involves higher health risks. Some of the consequences of high PM levels are already beginning to be noticed, such as the loss of visibility in the city at certain times of the day and premature deaths from PM have an average cost of 1\% of the country's gross domestic product \cite{itdp2019}.\\

Several temporal analyses of PM have been done and also studies trying to pinpoint the sources of this pollutant, either by chemical characterization and emissions inventories analysis. However, little research has been done taking into account the Aerosol Optical Depth, that has the advantage of measuring the whole vertical area of the atmosphere, and not only near the Earth surface. That’s why the aim of this study is to examine PM trends by comparing PM data measured on ground by SIMA, Aerosol Optical Depth (AOD) estimated by solar irradiance measurements and AOD data from NASA’s MODIS satellite. We hope to contribute to the better understanding of the behavior of the pollutant in the complex physiographic zone of the MMA, since it is limited by the Northern Gulf Coastal Plain in the northeast and the Sierra Madre Oriental in the southwest.\\

The MMA is an industrial power in Northeast Mexico and the third largest urban area in the country. Given this and the way the urban area developed, the distribution of the PM sources is not homogeneous. The MMA includes the municipalities of Apodaca, Cadereyta, Escobedo, García, Guadalupe, Juárez, Monterrey, San Nicolás de los Garza, San Pedro, Santa Catarina and begins to extend towards Santiago with an area of 7,657 km\textsuperscript{2} in 2015 \cite{inegi2015}. They all are limited by the Northern Gulf Coastal Plain in the northeast and the Sierra Madre Oriental in the southwest, which causes a downward flow of air from the mountains at night and in the early morning \cite{molina2019}. However, the predominant direction of the wind is from east to west and this transports the PM to the mountains, which acts as a barrier by accumulating the particles \cite{gonzalez2011}. A combination of the increase in wind speeds, high temperatures and humidity, is what makes PM concentrations drop in summer \cite{gonzalez2011,sima2019}.
