\subsubsection{OMI-NASA Ozone Data}
Se utilizaron los datos de la columna total de Ozono (TOC) en Dobson Units del satélite OMI de la NASA. Los datos fueron medidos diariamente con una resolución espacial de 0.25$^{\circ}$ Deg. Tomamos el producto OMI/Aura Ozone DOAS Total Column L3 1 day V3 (OMDOAO3e) \cite{OMI_ozone}, con un código escrito en \textit{Python} obtuvimos las mediciones diarias para el píxel correspondiente al área metropolitana de Monterrey, el valor que asigna el producto a cada pixel que no realizó una medición es $-1.267650600x10^{30}$, es por ello que a cada vez que se no se tenía una medición se le asignaba el promedio mensual del año correspondiente.
\subsubsection{OMI-NASA AOD Data}
Del producto OMI/Aura Near UV Aerosol Optical Depth and Single Scattering Albedo L3 1 day 1.0 degree x 1.0 degree V3 (OMAERUVd) \cite{OMI_AOD} se aprovecharon los datos del Aerosol Optical Depth 500nm (AOD500nm), con un código escrito en \textit{Python} se obtuvieron las mediciones diarias para el píxel correspondiente al área metropolitana de Monterrey.