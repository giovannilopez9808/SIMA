\begin{table}[H]
    % \begin{tabular}{llccccc} \hline
    %     Station name   & Code & Altitude (m.a.sl) & Latitude & Longitude & Operating since & Classification \\ \hline
    %     Obispado       & CE   & 560               & 25.67    & -100.34   & Sep/1992        & Mixed          \\
    %     Escobedo       & N    & 528               & 25.80    & -100.34   & Dec/2009        & Mixed          \\
    %     Universidad    & N2   & 520               & 25.73    & -100.31   & Oct/2017        &                \\
    %     San Nicolás    & NE   & 476               & 25.75    & -100.26   & Sep/1992        & Industrial     \\
    %     Apodaca        & NE2  & 432               & 25.78    & -100.19   & Jun/2011        & Industrial     \\
    %     San Bernabé    & NW   & 571               & 25.76    & -100.37   & Sep/1992        & Room           \\
    %     García         & NW2  & 716               & 25.78    & -100.59   & July/2009       & Industrial     \\
    %     Serena         & S    & 630               & 25.57    & -100.25   & Oct/2017        &                \\
    %     Pastora        & SE   & 492               & 25.67    & -100.25   & Sep/1992        & Room           \\
    %     Juárez         & SE2  & 387               & 25.65    & -100.10   & Nov/2012        & Room           \\
    %     Cadereyta      & SE3  & 340               & 25.36    & -100.00   & Aug/2017        &                \\
    %     Santa Catarina & SW   & 694               & 25.68    & -100.46   & Sep/1992        & Industrial     \\
    %     San Pedro      & SW2  & 636               & 25.66    & -100.41   & Feb/2014        & Room           \\ \hline
    % \end{tabular}
    \begin{tabular}{llcccc} \hline
        Station name   & Code & Altitude (m.a.sl) & Latitude & Longitude & Classification \\ \hline
        Obispado       & CE   & 560               & 25.67    & -100.34   & Mixed          \\
        Escobedo       & N    & 528               & 25.80    & -100.34   & Mixed          \\
        Universidad    & N2   & 520               & 25.73    & -100.31   &                \\
        San Nicolás    & NE   & 476               & 25.75    & -100.26   & Industrial     \\
        Apodaca        & NE2  & 432               & 25.78    & -100.19   & Industrial     \\
        San Bernabé    & NW   & 571               & 25.76    & -100.37   & Room           \\
        García         & NW2  & 716               & 25.78    & -100.59   & Industrial     \\
        Serena         & S    & 630               & 25.57    & -100.25   &                \\
        Pastora        & SE   & 492               & 25.67    & -100.25   & Room           \\
        Juárez         & SE2  & 387               & 25.65    & -100.10   & Room           \\
        Cadereyta      & SE3  & 340               & 25.36    & -100.00   &                \\
        Santa Catarina & SW   & 694               & 25.68    & -100.46   & Industrial     \\
        San Pedro      & SW2  & 636               & 25.66    & -100.41   & Room           \\ \hline
    \end{tabular}
    \caption{}
    \label{table:stations_loc}
\end{table}
We collected SIMA’s data from all 13 stations for the period of 2015-2020, where three of them started operating in 2017. Data includes hourly measures (CHECAR) of PM\textsubscript{10}, PM\textsubscript{2.5}, solar irradiance, humidity, wind and other pollutants. The measuring instruments that are located at SIMA’s meteorological station are required to adapt to the mexican guidelines for obtaining and communicating the air quality index and health risks that can be found as NOM-172-SEMARNAT-2019.
\begin{figure}[H]
    \centering
    \includegraphics[scale=0.15]{images/map.png}
    \caption{Incluir locación de MMA en México (climas), distribución, estaciones y alturas}
    \label{fig:map}
\end{figure}
\subsection{Meteorology, solar irradiance and PM\textsubscript{10} measurements}
De las estaciones San Nicolas y San Bernabé se realizó un filtro de días de cielo despejado haciendo uso de las mediciones de irradiancia solar en el rango 285nm a 2800nm. En la tabla \ref{table:measurements_SIMA} se muestra el porcentaje de mediciones anuales en cada estación de la red del SIMA, se relleno en rojo las estaciones que tienen un porcentaje menor a 75\%, ya que con esto se considera que las mediciones no son representativas \cite{molina2019}.
\begin{table}[H]
    \changefontsizes{9pt}
    \begin{tabular}{lcccccccccccc}
        \hline
        \multicolumn{1}{c}{}                       & \multicolumn{2}{c}{2015} & \multicolumn{2}{c}{2016} & \multicolumn{2}{c}{2017} & \multicolumn{2}{c}{2018} & \multicolumn{2}{c}{2019}                            & \multicolumn{2}{c}{2020}                                                                                                                                                                                                    \\ \cline{2-13}
        \multicolumn{1}{c}{\multirow{-2}{*}{Code}} & PM\textsubscript{10}     & SR                       & PM\textsubscript{10}     & SR                       & PM\textsubscript{10}                                & SR                                                  & PM\textsubscript{10} & SR   & PM\textsubscript{10} & SR                                                  & PM\textsubscript{10}                                & SR   \\ \hline
        CE                                         & 94.6                     & 95.8                     & 97.0                     & 97.7                     & 96.8                                                & 98.2                                                & 94.8                 & 99.1 & 95.1                 & 94.5                                                & 92.8                                                & 96.5 \\
        N                                          & 76.8                     & 98.8                     & 98.1                     & 97.9                     & 93.0                                                & 96.6                                                & 94.1                 & 97.6 & 96.3                 & 99.9                                                & \cellcolor[HTML]{CB0000}{\color[HTML]{FFFFFF} 73.7} & 84.3 \\
        N2                                         & -                        & -                        & -                        & -                        & \cellcolor[HTML]{CB0000}{\color[HTML]{FFFFFF} 20.9} & \cellcolor[HTML]{CB0000}{\color[HTML]{FFFFFF} 25.1} & 94.3                 & 98.6 & 94.4                 & 98.8                                                & 92.3                                                & 99.1 \\
        NE                                         & 93.9                     & 83.9                     & 98.3                     & 96.9                     & 98.1                                                & 99.2                                                & 97.4                 & 99.8 & 98.5                 & 99.5                                                & 92.8                                                & 89.3 \\
        NE2                                        & 89.9                     & 98.3                     & 93.7                     & 99.9                     & 96.3                                                & 99.8                                                & 96.1                 & 99.4 & 94.5                 & 99.1                                                & 91.1                                                & 98.1 \\
        NW                                         & 95.3                     & 99.3                     & 98.1                     & 94.9                     & 98.5                                                & 99.7                                                & 94.9                 & 99.5 & 90.9                 & 94.4                                                & 96.5                                                & 98.7 \\
        NW2                                        & 83.8                     & 97.7                     & 84.3                     & 88.5                     & 95.7                                                & 99.0                                                & 96.6                 & 99.5 & 95.8                 & 98.2                                                & 96.4                                                & 99.4 \\
        S                                          & -                        & -                        & -                        & -                        & \cellcolor[HTML]{CB0000}{\color[HTML]{FFFFFF} 22.8} & \cellcolor[HTML]{CB0000}{\color[HTML]{FFFFFF} 23.7} & 92.9                 & 99.4 & 96.7                 & 99.6                                                & 94.8                                                & 99.9 \\
        SE                                         & 92.8                     & 98.6                     & 97.2                     & 99.0                     & 98.4                                                & 99.1                                                & 94.3                 & 98.5 & 95.9                 & 98.4                                                & 94.1                                                & 99.1 \\
        SE2                                        & 95.9                     & 99.6                     & 99.1                     & 100.0                    & 95.7                                                & 97.4                                                & 92.4                 & 99.6 & 95.8                 & \cellcolor[HTML]{CB0000}{\color[HTML]{FFFFFF} 64.2} & 90.3                                                & 99.7 \\
        SE3                                        & -                        & -                        & -                        & -                        & \cellcolor[HTML]{CB0000}{\color[HTML]{FFFFFF} 37.1} & \cellcolor[HTML]{CB0000}{\color[HTML]{FFFFFF} 38.6} & 95.1                 & 99.1 & 98.3                 & 99.8                                                & 94.1                                                & 99.7 \\
        SW                                         & 92.3                     & 97.2                     & 96.9                     & 98.7                     & 97.9                                                & 99.6                                                & 92.6                 & 99.2 & 97.9                 & 99.6                                                & 97.0                                                & 99.6 \\
        SW2                                        & 88.4                     & 96.5                     & 96.8                     & 98.6                     & 93.6                                                & 98.1                                                & 95.1                 & 99.5 & 96.7                 & 99.7                                                & 90.8                                                & 99.9 \\ \hline
    \end{tabular}
    \caption{Porcentaje anual de las mediciones de PM\textsubscript{10} e irradiancia solar de las estacion del SIMA en el periodo 2015-2020.}
    \label{table:measurements_SIMA}
\end{table}
Se seleccionaron las estaciones de San Nicolas y San Bernabé ya que estas contaban con una mayor cantidad de datos disponibles y los radiómetros usados para las mediciones de irradiancia solar se sabían concretamente que miden en el rango 285-2800 nm. Realizando una visualización de los datos, se seleccionaron los días de cielo despejado usando el criterio () como los que se muetran en la figura \ref{fig:clear_days}.
\begin{figure}[H]
    \centering
    \includegraphics[scale=0.5]{images/Clear_days.png}
    \caption{}
    \label{fig:clear_days}
\end{figure}