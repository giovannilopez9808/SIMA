\subsection{Study Area}
El área metropolitana de Monterrey (MMA) se localiza en la parte noreste de México, con una población de 4.7 millones de personas distribuidas en 18 municipios, es la región urbana más poblada de México \cite{inegi2015}. Entre las principales actividades económicas son: el comercio, servicios inmobiliarios, construcción y fabricación de maquinaria y equipo que se realizan 213 grupos industriales, la mayoría con sede en el MMA. El MMA está rodeado por la Sierra Madre Oriental, el Cerro de la Silla, el Cerro de las Mitras y el Cerro del Topo Chico (figura \ref{fig:map}), las cuales crean una barrera física natural para la circulación del viento que impiden el desalojo del aire contaminado hacia el exterior de la zona.\cite{proaire2008}
\begin{figure}[H]
    \centering
    \includegraphics[scale=0.2]{images/map.png}
    \caption{Área Metropolitana de Monterrey con sus principales cerros y sierras.}
    \label{fig:map}
\end{figure}
El Sistema Integral de Monitoreo Ambiental (SIMA) inició sus operaciones en el año 1992 con cinco estaciones de monitoreo. En el año 2017 el número de estaciones en operación aumentó a 13, siendo las estaciones Universidad, Serena y Cadereyta las últimas instaladas en este año \cite{simapage}. La localización geográfica y clasificación de las estaciones de monitoreo se muestran en la Tabla \ref{table:stations_loc}. La clasificación de las estaciones corresponde con la descripción establecida por la Secretaría de Medio Ambiente y Recursos Natuales (SEMARNAT). \cite{proaire2016}
\begin{table}[H]
    % \begin{tabular}{llccccc} \hline
    %     Station name   & Code & Altitude (m.a.sl) & Latitude & Longitude & Operating since & Classification \\ \hline
    %     Obispado       & CE   & 560               & 25.67    & -100.34   & Sep/1992        & Mixed          \\
    %     Escobedo       & N    & 528               & 25.80    & -100.34   & Dec/2009        & Mixed          \\
    %     Universidad    & N2   & 520               & 25.73    & -100.31   & Oct/2017        &                \\
    %     San Nicolás    & NE   & 476               & 25.75    & -100.26   & Sep/1992        & Industrial     \\
    %     Apodaca        & NE2  & 432               & 25.78    & -100.19   & Jun/2011        & Industrial     \\
    %     San Bernabé    & NW   & 571               & 25.76    & -100.37   & Sep/1992        & Room           \\
    %     García         & NW2  & 716               & 25.78    & -100.59   & July/2009       & Industrial     \\
    %     Serena         & S    & 630               & 25.57    & -100.25   & Oct/2017        &                \\
    %     Pastora        & SE   & 492               & 25.67    & -100.25   & Sep/1992        & Room           \\
    %     Juárez         & SE2  & 387               & 25.65    & -100.10   & Nov/2012        & Room           \\
    %     Cadereyta      & SE3  & 340               & 25.36    & -100.00   & Aug/2017        &                \\
    %     Santa Catarina & SW   & 694               & 25.68    & -100.46   & Sep/1992        & Industrial     \\
    %     San Pedro      & SW2  & 636               & 25.66    & -100.41   & Feb/2014        & Room           \\ \hline
    % \end{tabular}
    \begin{tabular}{llcccc} \hline
        Station name   & Code & Altitude (m.a.sl) & Latitude & Longitude & Classification \\ \hline
        Obispado       & CE   & 560               & 25.67    & -100.34   & Mixed          \\
        Escobedo       & N    & 528               & 25.80    & -100.34   & Mixed          \\
        Universidad    & N2   & 520               & 25.73    & -100.31   &                \\
        San Nicolás    & NE   & 476               & 25.75    & -100.26   & Industrial     \\
        Apodaca        & NE2  & 432               & 25.78    & -100.19   & Industrial     \\
        San Bernabé    & NW   & 571               & 25.76    & -100.37   & Room           \\
        García         & NW2  & 716               & 25.78    & -100.59   & Industrial     \\
        Serena         & S    & 630               & 25.57    & -100.25   &                \\
        Pastora        & SE   & 492               & 25.67    & -100.25   & Room           \\
        Juárez         & SE2  & 387               & 25.65    & -100.10   & Room           \\
        Cadereyta      & SE3  & 340               & 25.36    & -100.00   &                \\
        Santa Catarina & SW   & 694               & 25.68    & -100.46   & Industrial     \\
        San Pedro      & SW2  & 636               & 25.66    & -100.41   & Room           \\ \hline
    \end{tabular}
    \caption{}
    \label{table:stations_loc}
\end{table}
\subsection{Meteorology, solar irradiance and PM\textsubscript{10} measurements}
En la Tabla \ref{table:measurements_SIMA} se muestra el porcentaje de mediciones anuales disponibles en cada estación de la red del SIMA. En rojo se representan las estaciones que tienen un porcentaje menor a 75\%, en estos casos se considera que el número de mediciones son insuficientes para una estadística anual \cite{molina2019}.
\begin{table}[H]
    \changefontsizes{9pt}
    \begin{tabular}{lcccccccccccc}
        \hline
        \multicolumn{1}{c}{}                       & \multicolumn{2}{c}{2015} & \multicolumn{2}{c}{2016} & \multicolumn{2}{c}{2017} & \multicolumn{2}{c}{2018} & \multicolumn{2}{c}{2019}                            & \multicolumn{2}{c}{2020}                                                                                                                                                                                                    \\ \cline{2-13}
        \multicolumn{1}{c}{\multirow{-2}{*}{Code}} & PM\textsubscript{10}     & SR                       & PM\textsubscript{10}     & SR                       & PM\textsubscript{10}                                & SR                                                  & PM\textsubscript{10} & SR   & PM\textsubscript{10} & SR                                                  & PM\textsubscript{10}                                & SR   \\ \hline
        CE                                         & 94.6                     & 95.8                     & 97.0                     & 97.7                     & 96.8                                                & 98.2                                                & 94.8                 & 99.1 & 95.1                 & 94.5                                                & 92.8                                                & 96.5 \\
        N                                          & 76.8                     & 98.8                     & 98.1                     & 97.9                     & 93.0                                                & 96.6                                                & 94.1                 & 97.6 & 96.3                 & 99.9                                                & \cellcolor[HTML]{CB0000}{\color[HTML]{FFFFFF} 73.7} & 84.3 \\
        N2                                         & -                        & -                        & -                        & -                        & \cellcolor[HTML]{CB0000}{\color[HTML]{FFFFFF} 20.9} & \cellcolor[HTML]{CB0000}{\color[HTML]{FFFFFF} 25.1} & 94.3                 & 98.6 & 94.4                 & 98.8                                                & 92.3                                                & 99.1 \\
        NE                                         & 93.9                     & 83.9                     & 98.3                     & 96.9                     & 98.1                                                & 99.2                                                & 97.4                 & 99.8 & 98.5                 & 99.5                                                & 92.8                                                & 89.3 \\
        NE2                                        & 89.9                     & 98.3                     & 93.7                     & 99.9                     & 96.3                                                & 99.8                                                & 96.1                 & 99.4 & 94.5                 & 99.1                                                & 91.1                                                & 98.1 \\
        NW                                         & 95.3                     & 99.3                     & 98.1                     & 94.9                     & 98.5                                                & 99.7                                                & 94.9                 & 99.5 & 90.9                 & 94.4                                                & 96.5                                                & 98.7 \\
        NW2                                        & 83.8                     & 97.7                     & 84.3                     & 88.5                     & 95.7                                                & 99.0                                                & 96.6                 & 99.5 & 95.8                 & 98.2                                                & 96.4                                                & 99.4 \\
        S                                          & -                        & -                        & -                        & -                        & \cellcolor[HTML]{CB0000}{\color[HTML]{FFFFFF} 22.8} & \cellcolor[HTML]{CB0000}{\color[HTML]{FFFFFF} 23.7} & 92.9                 & 99.4 & 96.7                 & 99.6                                                & 94.8                                                & 99.9 \\
        SE                                         & 92.8                     & 98.6                     & 97.2                     & 99.0                     & 98.4                                                & 99.1                                                & 94.3                 & 98.5 & 95.9                 & 98.4                                                & 94.1                                                & 99.1 \\
        SE2                                        & 95.9                     & 99.6                     & 99.1                     & 100.0                    & 95.7                                                & 97.4                                                & 92.4                 & 99.6 & 95.8                 & \cellcolor[HTML]{CB0000}{\color[HTML]{FFFFFF} 64.2} & 90.3                                                & 99.7 \\
        SE3                                        & -                        & -                        & -                        & -                        & \cellcolor[HTML]{CB0000}{\color[HTML]{FFFFFF} 37.1} & \cellcolor[HTML]{CB0000}{\color[HTML]{FFFFFF} 38.6} & 95.1                 & 99.1 & 98.3                 & 99.8                                                & 94.1                                                & 99.7 \\
        SW                                         & 92.3                     & 97.2                     & 96.9                     & 98.7                     & 97.9                                                & 99.6                                                & 92.6                 & 99.2 & 97.9                 & 99.6                                                & 97.0                                                & 99.6 \\
        SW2                                        & 88.4                     & 96.5                     & 96.8                     & 98.6                     & 93.6                                                & 98.1                                                & 95.1                 & 99.5 & 96.7                 & 99.7                                                & 90.8                                                & 99.9 \\ \hline
    \end{tabular}
    \caption{Porcentaje anual de las mediciones de PM\textsubscript{10} e irradiancia solar de las estacion del SIMA en el periodo 2015-2020.}
    \label{table:measurements_SIMA}
\end{table}
Se seleccionaron las estaciones de San Nicolás y San Bernabé ya que el radiómetro instalado Pyranometer Model 095 mide la irradiancia solar en el rango 285nm a 2800nm. Se generó una base de datos medidos bajo cielo despejado. El criterio de clasificación consistió en graficar la irradiancia solar a lo largo de las horas y sse selecciono sólo aquellas mediciones que forman una función gaussiana (figura \ref{fig:clear_days}). Debido a la extensión territorial no necesariamente las condicione meteorológicas son iguales al mismo tiempo. Esto provoca que la selección de días de cielo despejado varie para algunas estaciones.
\begin{figure}[H]
    \centering
    \includegraphics[scale=0.5]{images/Clear_days.png}
    \caption{Promedio de la irradiancia solar para cada epoca del año en las estaciones San Nicolás y San Bernabé.}
    \label{fig:clear_days}
\end{figure}
Los instrumentos y unidades de medición utilizadas para: la dirección y velocidad del viento, la precipitación y las partículas suspendidas menores a 10$\mu m$ (PM\textsubscript{10}) estan enlistados en la Tabla \ref{table:instruments}.
\begin{table}[H]
    \centering
    \begin{tabular}{lcc} \hline
        Parámetro            & Instrumento & Unidad de medición \\ \hline
        Dirección de viento  & Metone 24A  & Grados             \\
        Velocidad del viento & Metone 14A  & Km/h               \\
        Precipitación        & Metone 370  & mm/hr              \\
        PM$_{10}$            & BAM120      & $\mu g/m^3$        \\ \hline
    \end{tabular}
    \caption{Instrumentos de medición instalados en las estaciones de monitoreo del SIMA}
    \label{table:instruments}
\end{table}
La información recabada en los instrumentos es monitoreada continuamente y promediada para cada hora. La calibración y mantenimiento esta regida por los protocolos NOM-156-SEMARNAT-2012 \cite{SEMARNAT2012} y NOM-172-SEMARNAT-2019 \cite{SEMARNAT2019}.