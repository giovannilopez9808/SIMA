\subsubsection{Algortimo para determinar el AOD utilizando medicioness in situ}
Se escribió un código en el lenguaje \textit{Python}, el cual crea el archivo de input para el funcionamiento del modelo SMARTS. Utilizando la base de datos de días de cielo despejado que contiene el día, el mes y el año en formato yymmdd y la TOC correspondiente a las fechas clasificadas. El código escrito esta diseñado para ejecutar el modelo y obtener como resultado una matriz de espectros solares cada minuto entre las 9 y 16 horas. Posteriormente se calcula la irradiancia solar total con base a la ecuación \ref{eq:irradiance}.
\begin{equation}
    I(t) = \int\limits_{285}^{2800} E(\lambda,t) d\lambda
    \label{eq:irradiance}
\end{equation}
A partir de la obtención de la irradiancia solar minuto a minuto en función de las horas del día, se establece el valor máximo de irradiancia solar y se calcula el promedio centrado en 1 hora. La diferencia relativa entre el promedio de la irradiancia solar máxima medida y la irradiancia solar obtenida con el modelo usando la ecuación \ref{eq:rd}.
\begin{equation}
    RD = \left(\frac{Model-Measurement}{Measurement}\right)*100\%
    \label{eq:rd}
\end{equation}
Inicialmente el límite inferior y superior en la búsqueda del $AOD_{500}$ está entre 0 y 1 respectivamente $(AOD_i=0, AOD_f=1)$, el AOD de cada iteración en la búsqueda se calcula de la siguiente manera:
\begin{equation*}
    AOD=\frac{AOD_i+AOD_f}{2}
\end{equation*}
Dependiendo del valor de la diferencia relativa (RD) se calcula un valor al AOD y de los límites de la búsqueda. El criterio para determinar el AOD se basa en los siguientes puntos:
\begin{enumerate}
    \item Si $RD>11\%$, entonces el $AOD_i$ de la nueva búsqueda es el $AOD_i=AOD$.
    \item Si $RD<9\%$, entonces el $AOD_f$ de la nueva búsqueda $AOD_f=AOD$.
\end{enumerate}
Este proceso termina cuando se obtiene una diferencia relativa entre 9\% y 11\% y el valor de AOD se guarda en la base de datos de días de cielo despejado.