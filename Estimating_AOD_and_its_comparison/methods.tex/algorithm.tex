\subsubsection{Algortimo para determinar el AOD utilizando medicioness in situ}
Se escribió un código en el lenguaje \textit{Python}, el cual crea el archivo de input para el funcionamiento del modelo SMARTS, utilizando la base de datos de días de cielo despejado que contiene el día, el mes y el año (en formato yymmdd) y la TOC correspondiente. El modelo de búsqueda del AOD tiene un límite inferior y superior, al inicio se establece que estos límites son 0 y 1 $(AOD_{i,0}=0, AOD_{f,0}=1)$ respectivamente, el AOD de cada iteración en la búsqueda se calcula de la siguiente manera:
\begin{equation*}
    AOD_{n}=\frac{AOD_{i,n}+AOD_{f,n}}{2}
\end{equation*}
La búsqueda inicia con un AOD\textsubscript{0} de 0.5 , enseguida se ejecuta el modelo para obtener como resultado una matriz de espectros solares cada minuto entre las 9 y 16 horas. Posteriormente se calcula la irradiancia solar total con base a la ecuación \ref{eq:irradiance}.
\begin{equation}
    I(t) = \int\limits_{285}^{2800} E(\lambda,t) d\lambda
    \label{eq:irradiance}
\end{equation}
A partir de la irradiancia solar obtenida con el modelo, minuto a minuto en función de las horas del día, se establece el valor máximo de irradiancia solar y se calcula el promedio centrado en 1 hora. La diferencia relativa entre la irradiancia solar máxima medida y la irradiancia solar obtenida con el modelo, se calcula por medio de la ecuación \ref{eq:rd}.
\begin{equation}
    RD = \left(\frac{Model-Measurement}{Measurement}\right)*100\%
    \label{eq:rd}
\end{equation}
Dependiendo del valor de RD se calcula un nuevo valor de AOD y de los límites de la búsqueda. El criterio para determinar el AOD de la siguiente iteracción, se basa en las siguientes condiciones:
\begin{enumerate}
    \item Si $RD>11\%$, entonces $AOD_{i,n+1}=AOD_{n}$ y $AOD_{f,n+1}=AOD_{f,n}$.
    \item Si $RD<9\%$, entonces $AOD_{f,n+1}=AOD_{n}$ y $AOD_{i,n+1}=AOD_{i,n}$.
\end{enumerate}
Este proceso termina cuando se obtiene una diferencia relativa entre 9\% y 11\% y el valor de AOD se guarda en la base de datos de días de cielo despejado.