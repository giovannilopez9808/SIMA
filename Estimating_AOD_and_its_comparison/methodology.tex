\section{Methodology}
\subsection{Groud-level PM concentrations}
We collected SIMA’s data from all 13 stations for the period of 2015-2020, where three of them
started operating in 2017. Data includes hourly measures of PM\textsubscript{10}, PM\textsubscript{2.5},
solar irradiance, humidity, wind and other pollutants. The measuring instruments that are located at SIMA’s
meteorological station are required to adapt to the mexican guidelines for obtaining and communicating the
air quality index and health risks that can be found as NOM-172-SEMARNAT-2019.
\subsection{SMARTS AOD Model}
Only solar radiation measurements under clear skies were selected. The source code of the SMARTS model was
modified to enter the values of Table 1 and perform iterations with the AOD550nm, until the relative difference
between the measured and modeled irradiance reaches 10\% at solar noon.
\subsection{MODIS-NASA Data}
We automated the download of data of Aerosol Optical Depth at 550 nm (AOD\textsubscript{550nm}) from NASA’s
satellite MODIS (\url{https://ladsweb.modaps.eosdis.nasa.gov/}). The data is taken daily and with a spatial
resolution of 1 Deg. We used the product MODIS-Aqua Aerosol Cloud Water Vapor Ozone Daily L3 Global 1Deg CMG
(MYD08\_D3) and looked at two data sets: Deep Blue Combined Mean and Land and Ocean Mean. 
\subsection{Wind}
\subsection{Data processing}
We developed a code in Python to analyze the distinct data sets. For PM data, we analyzed availability,
trends, and selected the data that coincided with the time the satellite passed over the coordinates of
the stations. The AOD data were compared with the PM data to find the correlation. For the linear regression
the module sklearn.linear\_model was used.